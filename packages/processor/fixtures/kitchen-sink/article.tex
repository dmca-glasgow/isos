\documentclass{article}
\usepackage{amsmath}
\usepackage{amsthm}
\usepackage{float}
\usepackage{authblk}
\usepackage{orcidlink}
\usepackage[noabbrev, capitalise, nameinlink]{cleveref}
\usepackage{hyperref}
\usepackage[normalem]{ulem}
\usepackage{csquotes}
\usepackage{graphicx}
\usepackage{awesomebox}
\usepackage{minted}
\usepackage{snotez}

% \setsidenotes{footnote=false}

\theoremstyle{definition}
\newtheorem{theorem}{Theorem}[subparagraph]
\newtheorem{lemma}[theorem]{Lemma}
\newtheorem{corollary}[theorem]{Corollary}
\newtheorem{proposition}[theorem]{Proposition}
\newtheorem{conjecture}[theorem]{Conjecture}
\newtheorem{definition}{Definition}
\newtheorem{example}[definition]{Example}
\newtheorem{exercise}[definition]{Exercise}

\theoremstyle{remark}
\newtheorem{solution}[definition]{Solution}
\newtheorem{remark}[definition]{Remark}

\begin{document}

\title{A kitchen sink LaTeX Document}
\date{May 2025}
\author[1]{Alison Carefully \orcidlink{0000-0002-1825-0097}}
\author[2]{Ivor Question \orcidlink{0000-0002-1825-0097}}
\affil[1]{Department of Mathematics, University X}
\affil[2]{Department of Biology, University Y}
\maketitle

\begin{abstract}
Maecenas faucibus mollis interdum. Vivamus sagittis lacus vel augue laoreet rutrum faucibus dolor auctor. Donec sed odio dui. Integer posuere erat a ante venenatis dapibus posuere velit aliquet. Aenean eu leo quam. Pellentesque ornare sem lacinia quam venenatis vestibulum.

Lorem ipsum dolor sit amet, consectetur adipiscing elit. Integer posuere erat a ante venenatis dapibus posuere velit aliquet. Praesent commodo cursus magna, vel scelerisque nisl consectetur et. Vivamus sagittis lacus vel augue laoreet rutrum faucibus dolor auctor.
\end{abstract}

\section{Heading 2}
\subsection{Heading 3} \label{sec:bravo}
\section*{Heading 2 (unnumbered)}
\subsection{Heading 3}
\subsubsection{Heading 4}
\paragraph{Heading 5}
\subparagraph{Heading 6}

Refer to \cref{sec:bravo} for more information.

Some \emph{italic text}.

Alternative {\em italic text} syntax.

Alternative \textit{italic text} syntax.

Some \textbf{bold text}.

Some \emph{\textbf{bold and italic text}}.

Some\textsuperscript{superscript text}.

Some\textsubscript{subscript text}.

An endash: \textendash

An emdash: \textemdash

A url: \url{http://www.yahoo.com}

A link: \href{http://www.yahoo.com}{Yahoo}

Some \sout{strikethrough text}.

\begin{displayquote}
A quote block.
\end{displayquote}

Inline maths $x^2 - 5$ and display maths:

$$
x^2 - 5 x + 6 = 0
$$

Equations are numbered:

\begin{equation}
\label{eq:myref1}
x^2 - 5 x + 6 = 0
\end{equation}

\begin{equation}
x^2 - 5 x + 6 = 0
\end{equation}

Check out \autoref{eq:myref1} and \autoref{eq:myref2}.

Unordered list:

\begin{itemize}
\item alpha
\item bravo
\item charlie
\end{itemize}

Ordered list:

\begin{enumerate}
\item alpha
\item bravo
\item charlie
\end{enumerate}

Description list:

\begin{description}
\item[alpha] bravo
\item charlie
\item[delta] foxtrot
\end{description}

An image:

\includegraphics[alt={My alt text}]{logo.png}

A figure:

\begin{figure}[H]
  \label{fig:logo}
  \centering
  \includegraphics[alt={My alt text}]{logo.png}
  \caption{My \textbf{caption} text}
\end{figure}

Refer to \cref{fig:logo}.

\begin{theorem} Some text \end{theorem}
\begin{lemma} Some text \end{lemma}
\begin{corollary} Some text \end{corollary}
\begin{proposition} Some text \end{proposition}
\begin{conjecture} Some text \end{conjecture}
\begin{definition} Some text \end{definition}
\begin{example} Some text \end{example}
\begin{exercise} Some text \end{exercise}
\begin{solution} Some text \end{solution}
\begin{remark} Some text \end{remark}
\begin{proof} Some text \end{proof}

\begin{theorem} Some text \end{theorem}
\begin{lemma} Some text \end{lemma}
\begin{corollary} Some text \end{corollary}
\begin{proposition} Some text \end{proposition}
\begin{conjecture} Some text \end{conjecture}
\begin{definition} Some text \end{definition}
\begin{example} Some text \end{example}
\begin{exercise} Some text \end{exercise}
\begin{solution} Some text \end{solution}
\begin{remark} Some text \end{remark}
\begin{proof} Some text \end{proof}

A small table:

\begin{tabular}{lrr}
  1 & 2 & 3 \\
  Urban Clinic & 213 & 30 \\
  Rural Clinic & 156 & 43 \\
\end{tabular}

A full width table:

\begin{table}[H]
  \caption{My \emph{example} table caption}
  \begin{tabular}{lrr}
    Area & Number of dogs inspected & Number of dogs with a flea infestation \\
    Urban Clinic & 213 & 30 \\
    Rural Clinic & 156 & 43 \\
  \end{tabular}
  \label{tbl:example}
\end{table}

See \autoref{tbl:example}.

\notebox{I'm a \emph{note} callout box}
\tipbox{I'm a \emph{tip} callout box}
\warningbox{I'm a \emph{warning} callout box}
\cautionbox{I'm a \emph{caution} callout box}
\importantbox{I'm an \emph{important} callout box}

An example of \verb|inline code|.

A block of code:

\begin{verbatim}
code {
  display: inline-block;
  background: hsl(from #fff h s calc(l - 75));
  font-family: monospace;
  margin: 0 0.2em;
}
\end{verbatim}

Inline syntax-highlighted code: \mintinline{javascript}{var i = 0;}.

A block of syntax-highlighted code:

\begin{minted}{css}
code {
  display: inline-block;
  background: hsl(from #fff h s calc(l - 75));
  font-family: monospace;
  margin: 0 0.2em;
}
\end{minted}

And inline footnote \footnote{
  Text for \emph{footnote}.

  Paragraphs are ok.
}~and \footnote{Text for \emph{footnote}}.

Block footnotes \footnotemark[3] and \footnotemark[4].

Another paragraph.

\footnotetext[3]{
  Text for \emph{footnote}.

  Paragraphs are ok.
}

\footnotetext[4]{
  Text for \emph{footnote} 2.
}

Block footnotes with labels \footnotemark[alpha] and \footnotemark[bravo].

Another paragraph.

\footnotetext[alpha]{
  Text for \emph{footnote}.

  Paragraphs are ok.
}

\footnotetext[bravo]{
  Text for \emph{footnote} 2.
}

\end{document}
